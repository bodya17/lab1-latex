\documentclass[a4paper]{article}
\usepackage[utf8]{inputenc}
\usepackage[ukrainian]{babel}
\usepackage{amsmath}
\usepackage{amssymb} % for \blacktriangleleft \blacktriangleright

\begin{document}
	\subsection{Варіаційні задачі на умовний екстремум}

	\textbf{Означення.} Якщо на функції, від яких залежить функціонал, крім граничних, накладені ще й інші умови,
	які називають \textbf{\textit{зв’язками}}, то такі задачі називаються \textbf{варіаційними задачами на умовний екстремум.} \\
	Зв’язки поділяють на такі три види: 
	\begin{itemize}
		\item \textbf{скінченні (голономні):} $\varphi(x, y_1,\ldots, y_n) = 0;$

		\item \textbf{диференціальні (неголономні):}
			$\varphi(x, y_1, \ldots, y_n, y'_1, \ldots y'_n) = 0;$

		\item \textbf{інтегральні (ізопериметричні):} 
			$J[y] = \int_{a}^{b} \varphi(x, y, y')dx = l = const$.
	\end{itemize}

	\begin{center}
		\underline{\textbf{Задача Лагранжа}}
	\end{center}

	Дослідимо на екстремум функціонал
	\begin{equation}
		I[y_1, \ldots, y_n] = \int_{a}^{b} f(x, y_1, \ldots, y_n, y'_1, \ldots, y'_n)dx,
	\end{equation}
	що залежить від $n$ функцій $y_j(x)$, на які накладено умови (зв’язки)
	
	$$\varphi_i(x, y_1, \ldots, y_n) = 0, (i = \overline{1,m}, \quad m < n)$$
	і які задовольняють граничні умови
	$$y_j(a) = y_{ja}, \quad y_j(b) = y_{jb}, \quad (j=\overline{1,n}).$$
	
	Побудуємо функцію Лагранжа:
	\begin{equation}
		F = f(x, y_1, \ldots, y_n, y'_1, \ldots, y'_n) + 
		\sum_{i=1}^{m} \lambda_i(x) \varphi_i(x, y_1, \ldots, y_n),    			
	\end{equation}				      

	де $\lambda_i$ - функції від незалежної змінної $x$. Умовний екстремум функціонала $I$
	досягається на тих самих кривих, на яких реалізується безумовний екстремум функціонала
	$I^{*} = \int_{a}^{b} Fdx = \int_{a}^{b} \left(f + \sum_{i=1}^{m} \lambda_i \varphi_i \right)dx$.
	Для функціонала $I^{*}$ канонічна система рівнянь Ейлера має вигляд:
	
	\begin{equation}\label{eq:system_of_Euler_equations}
		\frac{\partial F}{\partial y_j} - \frac{d}{dx} \left(\frac{\partial F}{\partial y'_j}\right) = 0, (j = \overline{1, n}).
	\end{equation}

	До цих рівнянь додаємо умови зв’язку
	\begin{equation}
		\varphi_i(x, y_1, \ldots, y_n) = 0, (i = \overline{1, m}, \quad m < n)
	\end{equation}
	і отримуємо систему $m+n$ рівнянь з $m+n$ невідомими $ y_1, \ldots, y_n, \lambda_1, \ldots \lambda_m $.
	З граничних умов знаходимо значення $2n$ невідомих констант інтегрування.

	\underline{\textbf{Приклад 1.}} Знайти екстремум функціонала
	$$I[y_1, y_2] = \int_{0}^{\pi/2} (y_1^{2} + y_2^{2} - {y'_1}^{2} - {y'_2}^{2})dx$$
	за умови зв’язку $y_1 - y_2 - 2cos(x) = 0$ і граничних умов
	$y_1(0) = 1, \\ y_2(0) = -1,\quad y_1(\pi/2) = y_2(\pi/2) = 1$.
	
	$\blacktriangleright$ Запишемо функцію Лагранжа
	$$F = y_1^2 + y_2^2 - {y'_1}^{2} - {y'_2}^{2} + \lambda(x)(y_1 - y_2 - 2cosx).$$
	Знайдемо $\frac{\partial F}{\partial y_1} = 2y_1 + \lambda, \quad
		\frac{\partial F}{\partial y_2} = 2y_2 - \lambda, \quad
		\frac{\partial F}{\partial y'_1} = -2y'_1, \quad
		\frac{\partial F}{\partial y'_2} = -2y'_1$. 
	Тоді система \eqref{eq:system_of_Euler_equations} матиме вигляд
	$$\begin{cases}
		2y_1 + \lambda - \frac{d}{dx}\left(-2y'_1\right) = 0 \\
		2y_2 + \lambda - \frac{d}{dx}\left(-2y'_2\right) = 0 \\
	\end{cases}$$
	У результаті отримуємо систему 
	$$\begin{cases}
		2y_1 + \lambda(x) + 2y''_1 = 0, \\
		2y_2 + \lambda(x) + 2y_2^{''} = 0, \\
		y_1 - y_2 - 2cosx = 0,
	\end{cases}$$
	$$y_1(0) = 1,\quad y2(0) = -1,\quad y_1(\pi/2) = y_2(\pi/2) = 1.$$
	Додамо перші два рівняння і введемо нову невідому функцію $z$
	$$2(y_1 + y_2)^{''} + 2(y_1 + y_2) = 0, \quad y_1 + y_2 = z.$$
	Тоді маємо рівняння другого порядку із сталими коефіцієнтами $z'' + z = 0$,
	розв’язок якого $z = C_1cosx + C_2sinx$ або $y_1 + y_2 = C_1cosx + C_2sinx$.
	З граничних умов  знаходимо значення сталих інтегрування $C_1 = 0, C_2 = 2$. Отже,
	$y_1 + y_2 = 2 sinx$ і, додавши до нього умову зв’язку $y_1 - y_2 - 2cosx = 0$, остаточно
	знаходимо екстремаль
	$$\begin{cases}
		y_1 = cosx + sinx,\\
		y_2 = sinx - cosx.
	\end{cases}$$
	Тоді $f = y_1^{2} + y_2^{2} - {y'_1}^{2} -{y'_2}^{2} = (cosx + sinx)^2 + (sinx - cosx)^2 - $ 
		$$-(-sinx + cosx)^2 - (cosx + sinx)^2 = 0.$$
	Оскільки $F''_{y'_1y'_1} = -2 < 0,
	\begin{vmatrix}
		F''_{y'_1y'_1} & F''_{y'_1y'_2} \\
		F''_{y'_2y'_1} & F''_{y'_2y'_2} \\
	\end{vmatrix} =
	\begin{vmatrix}
		-2 & 0 \\
		0 & -2 \\
	\end{vmatrix} = 4 > 0 $, то на екстремалі
	$\begin{cases}
		y_1 = cosx + sinx  \\
		y_2 = sinx - cosx
	\end{cases}$ функціонал досягає сильного максимуму $I_{max}= 0.\blacktriangleleft$

	\begin{center}
		\underline{\textbf{Ізопериметричні задачі}}
	\end{center}
	
	У вузькому сенсі ізопериметричними називають задачі, в яких необхідно знайти геометричну фігуру
	максимальної площі при заданому периметрі.
	
	Нехай функції $f(x, y, y')$ і $\varphi(x, y, y')$
	мають неперервні частинні похідні другого порядку для $x \in [a,b]$ і будь-яких $y, y'$.
	Необхідно серед кривих $y = y(x) \in C^{1}[a, b]$ визначити ту, яка надає екстремуму функціоналу
	$I[y] = \int_{a}^{b} f(x, y, y')dx$
	і задовольняє граничні умови $y(a) = y_a$, $y(b) = y_b$
	та інтегральне рівняння зв’язку $J[y] = \int_{a}^{b} \varphi(x, y, y')dx = l = const$.\\
	\textit{\textbf{Теорема Ейлера.}}
		Якщо крива $y = y(x)$, яка задовольняє інтегральне рівняння зв’язку
		$J[y] = \int_{a}^{b} \varphi(x, y, y')dx = l = const$
		та граничні умови  $y(a) = y_a$, $y(b) = y_b$ і не є екстремаллю функціонала $J[y]$,
		надає екстремуму функціонала $I[y] = \int_{a}^{b} f(x, y, y')dx$,
		то існує така константа $\lambda$, що крива $y = y(x)$
		є екстремаллю функціонала $I^*[y] = \int_{a}^{b} F(x, y, y')dx$,
		де $F(x, y, y') = f(x, y, y') + \lambda \varphi(x, y , y')$ --- функція Лагранжа.\\
	\underline{\textbf{Закон взаємності ізопериметричних задач:}} Сукупність умовних екстремалей не залежить від того, чи шукати екстремум функціонала $I[y]$ для фіксованого значення функціонала $J[y]$ чи, навпаки, шукати екстремум для фіксованого значення $I[y]$.

	\underline{\textbf{Приклад 2.}} Знайти екстремум функціонала $\int_{0}^{1} {y'}^{2}dx$ за граничних умов
	$y(0) = 0,\quad y(1) = 5$ і обмеження $\int_{0}^{1} xydx = 1$.\\
	$\blacktriangleright$ Запишемо для функції Лагранжа $F = {y'}^{2} + \lambda x y $ рівняння Ейлера
	$$\lambda x - 2y'' = 0 \Rightarrow y = \frac{\lambda}{12} x^3 + C_1x + C_2.$$

\end{document}