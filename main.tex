\documentclass[a4paper]{article}
\usepackage[T1,T2A]{fontenc}
\usepackage[utf8]{inputenc}
\usepackage[ukrainian]{babel}
\usepackage{amsmath}
\usepackage{graphicx}
\begin{document}

6.5 Варіаційні задачі на умовний екстремум\\

Означення. Якщо на функції, від яких залежить функціонал, крім граничних, накладені ще й інші умови,
які називають зв’язками, то такі зада-чі називаються варіаційними задачами на умовний екстремум.
Зв’язки поділяють на такі три види: 
\begin{itemize}
\item \textbf{скінченні (голономні):} $\phi(x, y_1, ..., y_n)=0;$
\item \textbf{диференціальні (неголономні):}
	$\phi(x, y_1, \ldots, y_n, y^{'}_1,\ldots y^{'}_n,)=0;$
\item \textbf{інтегральні (ізопериметричні):}

\begin{center}
\textbf{Задача Лагранжа}
\end{center} 
\end{itemize}
\end{document}